\begin{abstract}
    Multi-Author Writing Style Analysis is a crucial task in computational linguistics and authorship attribution, aimed at identifying points of transition within a document. This task has significant applications in forensic linguistics, plagiarism detection, and content verification. Over the years, various approaches have been explored, ranging from traditional stylometric techniques to advanced deep learning models. Early research relied on lexical and syntactic features to detect style changes, but these methods faced limitations in handling nuanced transitions. Recent advancements, particularly the integration of transformer-based models, have significantly improved the accuracy of style change detection. This paper aims to provide an overview of the tasks, the datasets, the evaluation metrics, and the baseline models for the Multi-Author Writing Style Analysis Lab of the PAN track at CLEF 2025.
\end{abstract}