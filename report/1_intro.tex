\section{Introduction}

Writing style analysis is a fundamental problem in computational linguistics, with applications in authorship attribution, forensic text analysis, and collaborative content verification. A challenge in this domain is detecting when authorship changes within a multi-author document, a task known as style change detection. This problem has been the focus of multiple iterations of the PAN shared tasks, which have provided benchmark datasets and evaluation frameworks to advance the field.

Early approaches to style change detection leveraged handcrafted features such as n-grams, part-of-speech tag frequencies, and sentence structure analysis. These methods, while effective for basic segmentation tasks, struggled with complex, subtle changes in authorial style. The introduction of deep learning and pre-trained transformer models revolutionized the field, allowing for more nuanced analysis of writing styles. Models like BERT and DeBERTa have demonstrated superior performance in detecting stylistic shifts at both the paragraph and sentence levels.

This paper aims to explore existing methods and challenges in multi-author writing style analysis, reviewing state-of-the-art approaches such as deep learning models, transfer learning, and ensemble techniques. We will discuss the datasets, evaluation metrics, and baseline models used in the previous iterations of Multi-Author Writing Style Analysis Lab of the PAN track, ultimately attempting to improve the performance of style change detection.